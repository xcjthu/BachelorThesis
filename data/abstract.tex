% !TeX root = ../main.tex

% 中英文摘要和关键字

\begin{abstract}
  文档级别关系抽取旨在从一篇文档中抽取出该文档包含的所有的关系事实。该任务要求模型能够从多个句子中捕捉复杂的上下文语义信息。现有的文档级关系抽取工作主要关注于利用复杂的模型,从文档中抽取信息。但是很少有工作关注到大规模的文档级别远程监督数据。远程监督机制可以自动标注数据,使模型能够从更大规模的数据中学习更加通用的知识,更好地捕捉关系语义信息。远程监督机制的有效性在句子级别关系抽取中得到了充分的验证,但该机制在文档级别中将引入比句子级别更严重的错误标注问题,影响了模型训练。本文提出利用预训练的方式来利用远程监督数据,并提出了多个预训练任务,能够帮助模型更好地抽取实体特征、关系特征并区分正样例与负样例。实验结果表明,本文提出的预训练模型能够充分利用远程监督数据,在大规模手工标注的文档级关系抽取数据集上效果有明显的提升。


  \thusetup{
    keywords = {文档级别关系抽取;预训练;远程监督},
  }
\end{abstract}

\begin{abstract*}
  Document-level relation extraction (DocRE) aims to extract all possible relations from a long document, which requires models to capture complicated contextual information across multiple sentences. Existing approaches focus on utilizing various network architectures to capture relational semantics within documents. However, few works consider the large-scale distantly supervised data. Distant supervision mechanism provide a feasible approach to utilize more data. The effectiveness of the distant supervision mechanism has been proven in sentence-level relation extraction, but it will introduce more serious wrong labeling problem at the document level than at the sentence level. In this paper, we propose to make full use of large-scale distantly supervised data in DocRE via multiple pre-training tasks, which help the model to learn powerful representations in mention/entity-level and relation-level, and distinguish positive instances and negative instances. Experiments on the large-scale DocRE dataset show the effectiveness of our pre-trained model.
  \thusetup{
    keywords* = {document-level relation extraction, pre-training, distant supervision},
  }
\end{abstract*}
